\section{Vision}

\subsection{Chapter 1, 7, 11 (August 12)}

\subsubsection*{Discussions}
\begin{itemize}
\item Difference between perspective, weak perspective, orthographic (see table 1.1 on page 29)
\item Decomposition of projection matrix into calibration and rotation/translation
\item Essential and fundamental matrix derivations
\item Binocular reconstruction -- why only four constraints instead of six?
\item Rectification -- mainly makes finding correspondences easier, epipoles move to infinity
\item Difference between correlation vs. edge matching -- first one is correspondences between pixels, second one is higher level (between edges)
\item Multi-scale edge matching -- why second derivative filter? Because we want to match zero crossings (which are edges) to other zero crossings
\item Structure from motion: isn't equ 8.2 good enough? (Hard optimization problem, optimizing other things can be easier)
\item Uncalibrated weak-perspective - why does this work, since orthographic assumes points are on planes (it's because points are still in 3d, ortohgraphic is just the projection method)
\end{itemize}

\subsubsection*{Topics for Review}
\begin{itemize}
\item Perspective models
\item Camera intrinsics and extrinsics
  \begin{itemize}
    \item Intrinsics: center X/Y, scale X/Y, skew
    \item Extrinsics: rotation, translation
  \end{itemize}
\item Camera calibration and recovery of intrinsics/extrinsics
\item Essential/fundamental matrices
\item Binocular reconstruction
\item Local methods for binocular fusion (correlation between patches, multi-scale edge matching)
\item Global methods for binocular fusion (essentially find energy function, minimize it)
\item Structure from motion - set up constraints, solve nonlinear least squares
\item Weak calibration - estimate epipolar geometry, then do binocular stereopsis
\item Eight point algorithm - compute F, decompose to find E, decompose to find R and t
\item Uncalibrated weak-perspective - affine epipolar geometry/calibration
\item Uncalibrated perspective cameras, bundle adjustment
\end{itemize}


\subsection{Chapter 4, 9 (August 15)}

\subsubsection*{Discussions}
\begin{itemize}
\item Convolutions
\item A high level view of aliasing, nyquist frequency and low pass filtering
\item Image pyramids and use of scale representations
\item Watershed transform and mean-shift
\end{itemize}

\subsubsection*{Topics for Review}
\begin{itemize}
\item Table 4.1 (fourier transform chart)
\item Aliasing
\item Different clustering algorithms for segmentation
\end{itemize}
